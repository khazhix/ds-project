\section{Introduction}

A palindrome is a string that reads the same both ways.
Palindromic patterns appear in many research areas, from
formal language theory to molecular biology.

There are a lot of papers introducing algorithms and data structures
to facilitate different problems that involve palindromes.
One such data structure is EERTREE, a recently described
linear-sized palindromic tree introduced by Rubinchik \cite{RUBINCHIK2018249}.

In this project we aim to design at least one purely functional and
fully persistent version of a palindromic tree, implement it
in Haskell programming language and compare it
with other existing solutions. We are going to start with
a naive implementation and gradually arrive at an efficient
version, relying on some of the techniques described by
Okasaki \cite{Okasaki1998} for designing purely functional data structures.

We hope that purely functional variations will prove valuable
for some divide-and-conquer approaches to palindromic analysis.
We also believe that a fully persistent version might be useful
for comparative analysis of closely-related strings
(such as RNA string mutations).

Compared to earlier work, our new contributions are these:

\begin{itemize}
\item We introduce an efficient purely functional version of palindromic tree,
  and provide an implementation in Haskell, a non-strict, purely-functional
  programming language; we also prove that although adding symbols is $O(\log{}n)$ worst case, it is $\Theta(1)$ on average.
\item We provide a variation of that data structure that is more
  friendly to garbage collection, and demonstrate its efficiency
  by counting rich strings of length from $1$ to $n$ in $O(n)$ memory;
\item We design a double-ended variation with new $prepend$
  and $merge$ operations, and show that $merge$ can be implemented
  with $O(\sqrt{n})$ time and space on average;
\end{itemize}

\section{Problem: palindromic analysis}

\emph{Here we should describe the problem that we are trying to solve.}

\section{Palindromic tree}

\subsection{Naïve palindromic tree}

\subsection{Infinite palindromic tree}

\subsection{First applications}

\section{Memory efficiency}

\subsection{Counting rich strings}

\section{Double-ended palindromic tree}

\subsection{Prepending symbols}

\emph{Here we mention that prefixes and suffixes
in palindromes are the same, so we adjust data structure
to keep track of both without much changes and no
additional asymptotic complexity.}

\subsection{Merging palindromic trees}

\emph{Here describe an algorithm for efficient merge.}

\begin{lemma}
Set of palindromes for $S_1 + S_2$ is a union of
palindrome sets for $S_1$, $S_2$ and some new palindromes
that appear at the catenation point. These new palindromes
form consecutive descendants of palindromic suffixes of $S_1$
or palindromic prefixes of $S_2$.
\end{lemma}
\begin{proof}
  \emph{To be done by Timur Khazhiev.}
\end{proof}

\begin{lemma}
There is an algorithm that implements $merge$ in $\Theta(K)$,
where $K$ is the number of new palindromes in $S_1 + S_2$
compared to palindromes of $S_1$ and $S_2$ taken separately.
\end{lemma}
\begin{proof}
  \emph{To be done by Timur Khazhiev.}
\end{proof}

\begin{theorem}
$merge$ is $O(min(N, M, \sqrt{N+M}))$ on average
and $O(min(N, M))$ worst case, where $N$, $M$
are length of input strings.
\end{theorem}
\begin{proof}
  \emph{To be done by Timur Khazhiev.}
\end{proof}

\begin{proposition}
On average are $\Theta(1)$ new palindromes with different
centers after $merge$.
\end{proposition}
\begin{proof}
  \emph{To be done.}
\end{proof}

\begin{proposition}
It is possible to find all largest new palindromes with different
centers after $merge$ in $O(\log{} n + \log{} m)$ on average.
\end{proposition}
\begin{proof}
  \emph{To be done.}
\end{proof}

\section{Complexity Analysis}

Working with maximum palindromic suffixes and prefixes
is central to most operations with palindromic trees.
So to understand complexities of these operations we
study length of such suffixes and prefixes first.

\begin{lemma}
\label{lemma-palindrome-prob}
  A random string of length $n$ is a palindrome with probability $\sigma^{- \lfloor\frac{n}{2}\rfloor}$.
\end{lemma}
\begin{proof}
  For a random string to be a palindrome its first $\lfloor\frac{n}{2}\rfloor$
  symbols must match symbols from the right half exactly, leaving only
  $1$ valid candidate out of $\sigma^{\lfloor\frac{n}{2}\rfloor}$.
\end{proof}

\begin{theorem}
  Average length of the maximum palindromic suffix
  of a random string of length $n$ is $\Theta(1)$
  for alphabets of size $\sigma \ge 2$.
\end{theorem}
\begin{proof}
  Let $L_{avg}(n)$ be the average length of the maximum palindromic suffix
  of a random string of length $n$.
  $L_{avg}(n) \ge 0$, so to prove $L_{avg}(n)$ is $\Theta(1)$ we need
  to find a constant upper bound.

  Let $P_{max}(k, n)$ be the probability that a string of length $n$
  has maximum palindromic suffix of length $k$.
  Then $L_{avg}(n)$ can be expressed as a weighted sum:
  $$
  L_{avg}(n) = \sum_{k=1}^{n} k \cdot P_{max}(k, n)
  $$
  We notice that for a string to have a maximum palindromic suffix
  of length $k$ it is necessary that last $k$ symbols form a palindrome.
  Using Lemma~\ref{lemma-palindrome-prob} together with that observation
  we get an upper bound of $P_{max}(k, n)$:
  $$
  P_{max}(k, n) \le \sigma^{- \lfloor\frac{k}{2}\rfloor}
  $$
  This allows us to derive an upper bound for $L_{avg}(n)$:
  \begin{align*}
    L_{avg}(n)
      &\le \sum_{k=1}^{n} k \cdot \sigma^{- \lfloor\frac{k}{2}\rfloor}
       \le \sum_{k=1}^{n} k \cdot \sigma^{- \frac{k-1}{2}}
       \le \sum_{k=1}^{\infty} k \cdot \sigma^{- \frac{k-1}{2}} \\
      &= \frac{\sigma}{(\sqrt{\sigma} - 1)^2}
       = O(1)
  \end{align*}
\end{proof}

\begin{corollary}
  $L_{avg}(n) < 12$ for $\sigma = 2$.
\end{corollary}

\begin{corollary}
  $L_{avg}(n) \le 4$ for $\sigma = 4$.
\end{corollary}

\begin{theorem}
  Adding a symbol to the end of the eertree of
  a string of length $n$ is $\Theta(1)$ on average,
  but $O(\log{}n)$ worst case.
\end{theorem}
\begin{proof}
  \emph{To be done by Timur Khazhiev.}
\end{proof}

\section{Related and future work}

\section{Conclusion}


%
% The first command in your LaTeX source must be the \documentclass command.
\documentclass[sigconf]{acmart}
\usepackage{amsmath}

    
%
% end of the preamble, start of the body of the document source.
\settopmatter{printacmref=false} % Removes citation information below abstract
\renewcommand\footnotetextcopyrightpermission[1]{} % removes footnote with conference information in first column
\pagestyle{plain} % removes running headers
\begin{document}

%
% The "title" command has an optional parameter, allowing the author to define a "short title" to be used in page headers.
\title{Processing palindromes with EERTREE data structure.}

%
% The "author" command and its associated commands are used to define the authors and their affiliations.
% Of note is the shared affiliation of the first two authors, and the "authornote" and "authornotemark" commands
% used to denote shared contribution to the research.
\author{Timur Khazhiev}
\email{t.khazhiev@innopolis.ru}
\affiliation{%
  \institution{Innopolis University}
}

\author{Nikolai Kudasov as a Supervisor}
\email{n.kudasov@innopolis.ru}
\affiliation{%
  \institution{Innopolis University}}


%
% The abstract is a short summary of the work to be presented in the article.
\begin{abstract}
This document represents a Data Science project proposal. It contains description of the project, the problem it required to solve, motivation and brief plan of execution of the project.
\end{abstract}


%
% This command processes the author and affiliation and title information and builds
% the first part of the formatted document.
\maketitle

\section{Introduction}
For this course EERTREE was chosen. EERTREE or palindromic tree is an efficient data structure for processing palindromes in strings. During this course I will be working on analysis and research on this data structure especially from functional programming paradigm point of view.

\section{Work Plan}
\par
\textbf{Iteration I. 11.02 - 25.02}\\
Analyze palindromic tree and similar solutions, algorithms and data structures. Compare them and find pros and cons.\\

\textbf{Iteration II. 25.02 - 11.03}\\
Come up with compromise solution (or many solutions). Start of  implementation. \\

\textbf{Iteration III. 11.03 - 25.03}\\
Implement known solution and complete own solution. Compare them: show results for different use cases (asymptotically and/or real time/memory usage, theoretical assurance via prover).   \\

\textbf{Iteration IV. 25.03 - 08.04}\\
Working with one or more of the options:
\begin{itemize} 
\item Partial or full persistence
\item Real-time performance
\item Stream fusion implementation
\item Cache oblivious model optimization
\item Modify/extend features
\end{itemize} 
\hfill

\textbf{Iteration V. 08.04 - 22.04} \\
Working with one or more of the options:
\begin{itemize} 
\item Partial or full persistence
\item Real-time performance
\item Stream fusion implementation
\item Cache oblivious model optimization
\item Modify/extend features
\end{itemize}
\hfill \break
\textbf{Iteration VI. 22.04 - 06.05} \\
Finalization of project: writing tests, formal proving, refactoring. Sum up results.
\end{document}

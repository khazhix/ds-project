\section{Introduction}

A palindrome is a string that reads the same both ways.
Palindromic patterns appear in many research areas, from
formal language theory to molecular biology.

There are a lot of papers introducing algorithms and data structures
to facilitate different problems that involve palindromes.
One such data structure is EERTREE, a recently described
linear-sized palindromic tree introduced by Rubinchik \cite{RUBINCHIK2018249}.

In this project we aim to design at least one purely functional and
fully persistent version of a palindromic tree, implement it
in Haskell programming language and compare it
with other existing solutions. We are going to start with
a naive implementation and gradually arrive at an efficient
version, relying on some of the techniques described by
Okasaki \cite{Okasaki1998} for designing purely functional data structures.

We hope that purely functional variations will prove valuable
for some divide-and-conquer approaches to palindromic analysis.
We also believe that a fully persistent version might be useful
for comparative analysis of closely-related strings
(such as RNA string mutations).

\section{Work Plan}

\subsection{Iteration I (Feb 11 - Feb 25)}

Analyse EERTREE data structure closely.
Collect common palindrome-related problems.
Research alternative algorithms and data structures for those problems.
Investigate existing approaches to palindromic analysis.
Form a list of important operations on EERTREE
to compare for time and space complexity.

\subsection{Iteration II (Feb 25 - Mar 11)}

Propose a purely functional, fully persistent palindromic tree.
Analyse time and space complexities for the most important operations
(both worst-case and amortized).
Compare with the original palindromic tree.

\subsection{Iteration III (Mar 11 - Mar 25)}

Implement EERTREE and the purely functional palindromic tree.
Compare implementations on sample problems.
Analyse purely functional implementation performance.

\subsection{Iteration IV (Mar 25 - Apr 08)}

Investigate options to improve performance for
the purely functional palindromic tree.
Consider these options:

\begin{itemize}
  \item fusion/deforestation optimisations;
  \item cache-oblivious model optimizations;
  \item linked vs. vector-based implementations;
  \item asymptotic improvements via partial evaluation;
  \item optimisations for small alphabet strings.
\end{itemize}

\subsection{Iteration V (Apr 08 - Apr 22)}

Design a generalised interface to the data structure,
relying on parametric polymorphism and higher-order functions
to facilitate reusability.

\subsection{Iteration VI (Apr 22 - May 06)}

Finalise the project by creating a Cabal package complete with
examples, tests, code comments, documentation and, perhaps, a
formal verification.

\section{Iteration results}

\subsection{Iteration I}
During iteration I github repository\cite{khazhix} was created where all related to the topic materials were collected. In general this iteration was successful: the analysis of data structure revealed a lack of functionality, interesting problems was found and overview with comparison of similar approaches was made. Detailed results for this iteration can be found in separate pdf file\cite{khazhix1} in github repository.
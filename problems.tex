\documentclass[12pt]{article}
\usepackage{fullpage}
\usepackage{float}
\usepackage{indentfirst}
\usepackage{minted}
\usepackage{graphicx}		%% to insert pics
\usepackage{listings}
\usepackage{subcaption}
\usepackage{hyperref}
\usepackage{color}
\usepackage{amsmath}
\usepackage{mathtools}
\usepackage{amssymb}
\newcommand*{\Z}{\mathbb{Z}}

\title{List of problems for eertree}
\author{Timur Khazhiev (\href{mailto:t.khazhiev@innopolis.ru}{t.khazhiev@innopolis.ru})}

\definecolor{mygreen}{rgb}{0,0.6,0}
\definecolor{mygray}{rgb}{0.5,0.5,0.5}
\definecolor{mymauve}{rgb}{0.58,0,0.82}

\lstset{ %
  backgroundcolor=\color{white},   % choose the background color
  basicstyle=\footnotesize,        % size of fonts used for the code
  breaklines=true,                 % automatic line breaking only at whitespace
  captionpos=b,                    % sets the caption-position to bottom
  commentstyle=\color{mygreen},    % comment style
  escapeinside={\%*}{*)},          % if you want to add LaTeX within your code
  keywordstyle=\color{blue},       % keyword style
  stringstyle=\color{mymauve},     % string literal style
}

\begin{document}

\maketitle


\section*{Problem 1}
 Consider an arbitrary string $g$. We will call this string as palindrome generator. The set of palindromes $P(g)$ that are generated by this string is determined as follows.

Let string length be n. For all $i$ from 1 to $n$ in $P(g)$ strings $g[1..i]g[1..i]^r$ and $g[1..i]g[1..i-1]^r$ are included, where $\alpha^r$ means $\alpha$, written in reversed order.

For example if $g="olymp"$, then $P(g)=\{"oo", "o", "ollo", "olo", "olyylo", "olylo", $ $"olymmylo", "olymylo", "olymppmylo", "olympmylo"\}$.

For a given generator of palindromes $g$ and the string $s$, it is required to find the number of occurrences of string from $P(g)$ in $s$ as substrings. Namely, it is required to find the number of pairs $(i, j)$ such that $s[i..j] \in P(g)$.
\par
\begin{table}[H]
\begin{tabular}{|l|l|}
\hline
Example input                                                            & Example output \\ \hline
\begin{tabular}[c]{@{}l@{}}olymp\\ olleolleolympmyolylomylo\end{tabular} & 7              \\ \hline
\end{tabular}
\end{table}
\href{https://www.e-olymp.com/en/problems/2468}{\underline{Link to a problem}}
\section*{Problem 2}
Each palindrome can be always created from the other palindromes, if a single character is also a palindrome. For example, the string $"malayalam"$ can be created by some ways:
\begin{align*} 
& malayalam = m + ala + y + ala + m\\
& malayalam = m + a + l + aya + l + a + m
\end{align*}


We want to take the value of function $NumPal(s)$ which is the number of different palindromes that can be created using the string $S$ by the above method. If the same palindrome occurs more than once then all of them should be counted separately.
\begin{table}[H]
\begin{tabular}{|l|l|}
\hline
Example input & Example output \\ \hline
malayalam     & 15             \\ \hline
\end{tabular}
\end{table}
\href{https://www.spoj.com/problems/NUMOFPAL/}{\underline{Link to a problem}}
\section*{Problem 3}
For a given string $S$ we want to find minimum number of continuous palindromes in which this string can be broken. 
\begin{table}[H]
\begin{tabular}{|l|l|}
\hline
Example input  & Example output \\ \hline
abacdc         & 2              \\ \hline
ababa          & 1              \\ \hline
ababbacababbad & 5              \\ \hline
abcd           & 4              \\ \hline
\end{tabular}
\end{table}
\href{https://www.spoj.com/problems/IITKWPCE/}{\underline{Link to a problem}}
\section*{Problem 4}
For a given string $S$ we want to find total number of continuous palindromes in which this string can be broken.
\begin{table}[H]
\begin{tabular}{|l|l|}
\hline
Example input & Example output \\ \hline
aaa           & 6              \\ \hline
aba           & 4              \\ \hline
\end{tabular}
\end{table}
\href{https://informatics.mccme.ru//mod/statements/view.php?chapterid=1750}{\underline{Link to a problem}}
\section*{Problem 5}
For a given string $s$ we want to find non palindrome of length $n$ between 2 palindromes of length $m$. Can be practical for searching potential hairpins.
\begin{table}[H]
\begin{tabular}{|l|l|}
\hline
Example input & Example output                                          \\ \hline
aaabcdaaa     & \begin{tabular}[c]{@{}l@{}}aaa\\ bcd\\ aaa\end{tabular} \\ \hline
\end{tabular}
\end{table}
\section*{Problem 6}
For a given string $s$ we want to find closest palindrome by edit distance.
\begin{table}[H]
\begin{tabular}{|l|l|}
\hline
Example input & Example output \\ \hline
abcda         & abcba          \\ \hline
abab          & ababa          \\ \hline
\end{tabular}
\end{table}
\end{document}
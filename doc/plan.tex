\documentclass[sigconf]{acmart}

\usepackage{booktabs} % For formal tables


% Copyright
\setcopyright{none}
%\setcopyright{acmcopyright}
%\setcopyright{acmlicensed}
%\setcopyright{rightsretained}
%\setcopyright{usgov}
%\setcopyright{usgovmixed}
%\setcopyright{cagov}
%\setcopyright{cagovmixed}

\settopmatter{printacmref=false} % Removes citation information below abstract
\renewcommand\footnotetextcopyrightpermission[1]{} % removes footnote with conference information in first column

\begin{document}
\title{Processing palindromes with EERTREE data structure}
%\titlenote{Produces the permission block, and
%  copyright information}
%\subtitle{Extended Abstract}
%\subtitlenote{The full version of the author's guide is available as
%  \texttt{acmart.pdf} document}


\author{Timur Khazhiev}
\affiliation{%
  \institution{Innopolis University}
}
\email{t.khazhiev@innopolis.ru}

\author{Nikolai Kudasov}
\authornote{Project supervisor.}
\affiliation{%
  \institution{Innopolis University}
}
\email{n.kudasov@innopolis.ru}

\begin{abstract}
  This document represents a Data Science project proposal.
  It contains description of the project, the problem it required to solve,
  motivation and brief plan of execution of the project.
\end{abstract}

%
% The code below should be generated by the tool at
% http://dl.acm.org/ccs.cfm
% Please copy and paste the code instead of the example below.
%
\begin{CCSXML}
<ccs2012>
<concept>
<concept_id>10003752.10003809.10010031</concept_id>
<concept_desc>Theory of computation~Data structures design and analysis</concept_desc>
<concept_significance>500</concept_significance>
</concept>
</ccs2012>
\end{CCSXML}

\ccsdesc[500]{Theory of computation~Data structures design and analysis}

\keywords{palidrome, eertree, purely functional, persistent}

\maketitle

\section{Introduction}
For this course EERTREE was chosen. EERTREE or palindromic tree is an efficient data structure for processing palindromes in strings. During this course I will be working on analysis and research on this data structure especially from functional programming paradigm point of view.

\section{Work Plan}
\textbf{Iteration I. 11.02 - 25.02}\\
Analyze palindromic tree and similar solutions, algorithms and data structures. Find problems it solves. Define set of interesting operations over palindromic tree.\\

\textbf{Iteration II. 25.02 - 11.03}\\
Propose functional (persistent) version of palindromic tree. Analyze theoretical complexity (worst case and amortized) and compare it with original solution.\\

\textbf{Iteration III. 11.03 - 25.03}\\
Implement functional and original version. Compare them: show results for different use cases (benchmarks, real time/memory usage).\\

\textbf{Iteration IV. 25.03 - 08.04}\\
Working with one or more of the options:
\begin{itemize} 
\item Stream fusion implementation
\item Cache oblivious model optimization
\item Linked and Array (vector) based implementation comparison
\end{itemize} 
\ 

\textbf{Iteration V. 08.04 - 22.04} \\
Implement convenient interface, generalization.
\\ \par
\textbf{Iteration VI. 22.04 - 06.05} \\
Finalization of project: writing tests, documentation, formal proving, refactoring. Cabal contribution. Sum up results.


\bibliographystyle{ACM-Reference-Format}
\bibliography{plan}

\end{document}

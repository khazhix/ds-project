\section{Introduction}
For this course EERTREE was chosen. EERTREE or palindromic tree is an efficient data structure for processing palindromes in strings. During this course I will be working on analysis and research on this data structure especially from functional programming paradigm point of view.

\section{Work Plan}
\textbf{Iteration I. 11.02 - 25.02}\\
Analyze palindromic tree and similar solutions, algorithms and data structures. Find problems it solves. Define set of interesting operations over palindromic tree.\\

\textbf{Iteration II. 25.02 - 11.03}\\
Propose functional (persistent) version of palindromic tree. Analyze theoretical complexity (worst case and amortized) and compare it with original solution.\\

\textbf{Iteration III. 11.03 - 25.03}\\
Implement functional and original version. Compare them: show results for different use cases (benchmarks, real time/memory usage).\\

\textbf{Iteration IV. 25.03 - 08.04}\\
Working with one or more of the options:
\begin{itemize} 
\item Stream fusion implementation
\item Cache oblivious model optimization
\item Linked and Array (vector) based implementation comparison
\end{itemize} 
\ 

\textbf{Iteration V. 08.04 - 22.04} \\
Implement convenient interface, generalization.
\\ \par
\textbf{Iteration VI. 22.04 - 06.05} \\
Finalization of project: writing tests, documentation, formal proving, refactoring. Cabal contribution. Sum up results.
